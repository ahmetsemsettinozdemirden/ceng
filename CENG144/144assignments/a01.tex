\documentclass[11pt]{article}
%Increase the text height
\addtolength{\voffset}{-62pt}
\addtolength{\textheight}{62pt}


%Increase the text width
\addtolength{\hoffset}{-22pt}
\addtolength{\oddsidemargin}{-32pt}
\addtolength{\marginparsep}{-11pt}
\addtolength{\marginparwidth}{-45pt}
\addtolength{\textwidth}{110pt}


\begin{document}
\pagestyle{myheadings}
\markright{\sc id: Name Surname%
\hfill Math 144 HW01 /p.}

\paragraph{Q1.} Consider the following argument:
\begin{enumerate}
\item if a student's CENG115 grade is CB the student will pass Finite Mathematics;
\item Ali is a student who took CENG115;
\item Ali's grade in CENG115 grade is DD;
\item therefore, Ali will not pass Finite Mathematics.
\end{enumerate}
Trust us: the above argument is \emph{very wrong}! Find and explain the error
in the argument (Hint: \emph{inverse error})

\paragraph{Q2.} Suppose that a student misses the first midterm, obtains 20 on
the second midterm and misses the final. Suppose further the same student
presents a valid doctors note for missing the final exam and receives 25 on
the make-up exam. What homework grade does the student need to pass the
course.

\paragraph{Q3.} The homework grades of a student in Math 144 are
\begin{center}
\begin{tabular}{*{14}{c}}
H01 & H02 & H03 & H04 & H05 & H06 & H07 & H08 & H09 
	& H10 & H11 & H12 & H13 & H14
\\\hline
10 & 12 & 1 & 5 & 3 & 15 & 15 & & & & 10 & 15 & 15
\end{tabular}
\end{center}
How many points will the homeworks contribute towards the student's letter
grade?

\paragraph{Q4.} For each linear equation bellow describe the set of solutions.
If the equation is not linear explain why.
\begin{enumerate}
\item \(2 x_1 + x_2(1+x_3) + \sqrt{4}x_3= 4\)
\item \((\cos 2) x_1 + \ln (7x_2) + \sqrt{4}x_3= 4\)
\item \(0 x_1 + (\ln e^6) x_2 + \sqrt{4}x_3= 4^{\log_4 7}\)
\end{enumerate}

\paragraph{Q5. } Consider the following system of linear equations in
\(\{x_1,x_2,\dots,x_9\}\).
\begin{eqnarray*}
3 x_{1} -  x_{2} + x_{8} - 5 x_{9} &=& 1
\\x_{3} + 2 x_{5} + x_{6} + 3 x_{7} &=& 3
\\2 x_{2} + 5 x_{4} + 2 x_{5} + x_{9} &=& 4
\\2 x_{2} + 2 x_{5} + x_{9} &=& 9
\end{eqnarray*}
which of the following is a solution to the above system of linear equation:
\begin{enumerate}
\item \(\left(1,\,2,\,-1,\,-1,\,2,\,3,\,-1,\,5,\,1\right)\)
\item \(\left(1,\,2,\,-1,\,-1,\,2,\,0,\,-1,\,5,\,1\right)\)
\item \(\left(1,\,2,\,-1,\,-1,\,2,\,3,\,-1,\,5,\,1,\, 0\right)\)
\end{enumerate}


\end{document}
